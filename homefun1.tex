\documentclass{article}

\usepackage{listings}
\usepackage{graphicx}
\usepackage{amsmath}

\title{Homefun 1}
\author{Ben Gamari}

\begin{document}
\maketitle

\section{Question 1}

We are given a network of nodes with edges $E = \{ oa, ob, oc, ab, ad, cd,
be, ce, dn, en \}$, with maximum flow rates $c_e \forall e \in E$. We want
to determine the direction and magnitude of the flow demanded of each
edge such that the flow between $o$ and $n$ is maximized.

The variables are the flow demanded of each edge, $f_e$ such that
$|f_e| \le c_e$. However, to conveniently cast this as in our
optimization framework (which expects configuration vectors to be
positive semidefinite (PSD)), we can rewrite this as $f'_e = f_e + c_e$ such
that $0 \le f'_e \le 2 c_e$.

We demand that the net flow in to each node is zero,

\begin{align*}
  f_{oa} - f_{ab} - f_{ad} = 0 \tag{node $a$} \\
  f_{ob} + f_{ab} - f_{be} = 0 \tag{node $b$} \\
  f_{oc} - f_{cd} - f_{ce} = 0 \tag{node $c$} \\
  f_{ad} + f_{cd} - f_{dn} = 0 \tag{node $d$} \\
  f_{be} + f_{ce} - f_{en} = 0 \tag{node $e$} \\
\end{align*}

Which we can then rewrite in terms of $f'$,

\begin{align*}
  f'_{oa} - f'_{ab} - f'_{ad} = 1  \tag{node $a$} \\
  f'_{ob} + f'_{ab} - f'_{be} = -1 \tag{node $b$} \\
  f'_{oc} - f'_{cd} - f'_{ce} = -7 \tag{node $c$} \\
  f'_{ad} + f'_{cd} - f'_{dn} = 1  \tag{node $d$} \\
  f'_{be} + f'_{ce} - f'_{en} = 6  \tag{node $e$} \\
\end{align*}

Which is of form $A \vec f' = b$ where $A$ is read off of the
left-hand side and $b$ from the right.

Since we demand that the net flow in each node (other than $o$ and
$n$) is zero, we know that the net flow into $n$ is precisely equal to
the net flow out of $o$. Consequently, we can take either of these net
flows to be our objective function,

\begin{align*}
  F(\vec f) & = f_{dn} + f_{en} = - f_{oa} - f_{ob} - f_{oc} \\
            & = f'_{dn} + f'_{en} + \mathrm{const} \\
            & = - f'_{oa} - f_{ob} - f_{oc} + \mathrm{const} \\
\end{align*}

Again, this is in the form of $c^T \vec f'$ (up to constants which
can be ignored in the optimization problem).

After solving our optimization problem,

\[ \mathrm{min}_{\vec f'} ~ c^T \vec f' ~ \mathrm{s.t.} ~ A \vec f', \vec f' \ge 0 \]

We can extract our flows $f_e = f'_e - c_e$.

Mathematica tells us (where, e.g., {\tt foc} corresponds to $f'_{oc}$),

\begin{lstlisting}
In[11]:= Minimize[{
  fdn + fen,
  foa - fab - fad == 1
   && fob + fab - fbe == -1
   && foc - fcd - fce == -7
   && fad + fcd - fdn == 1
   && fbe + fce - fen == 6
   && foa >= 0 && fob >= 0 && foc >= 0
   && fab >= 0 && fad >= 0 && fcd >= 0
   && fbe >= 0 && fce >= 0 && fdn >= 0 && fen >= 0
  }, {foa, fob, foc, fab, fad, fcd, fbe, fce, fdn, fen}]

Out[11]= {1, {foa -> 1, fob -> 0, foc -> 0, fab -> 0, fad -> 0,
  fcd -> 2, fbe -> 1, fce -> 5, fdn -> 1, fen -> 0}}
\end{lstlisting}

Resulting in a bandwidth of 5 units.

\section{Question 2}

We define the configuration space of this problem to consist of the
following variables,

\begin{itemize}
  \item $x_{ij}$, the quantity each raw component $i$ used to produce blend $j$

  \item $y_i$, the quantity of each raw component $i$ to buy

  \item $z_j$, the quantity of each blend $j$ to produce and sell
\end{itemize}
%
We denote the quantities given in the problem description with the following,

\begin{itemize}
\item $b_i$, the quantity of raw component $i$ available for purchase
\item $d_j$, the maximum demand of blend $j$
\item $p_i$, the purchase price of raw component $i$
\item $p'_j$, the selling price of blend $j$
\item $e_i$, the selling price of excess of raw component $i$
\item $o_i$, the octane content of raw component $i$
\item $r_j$, the minimum octane content of blend $j$
\end{itemize}

We note that the amount of raw components $x_{ij}$ used to produce a
blend is related to the amount of the blend produced (under an
assumption of volume conservation),

\[ \sum_i x_{ij} = z_j \tag{C1} \]

The constraint on the availability of raw component $i$ can be written
as,
\[ \sum_j x_{ij} \le b_i \]
This inequality can be recast as an equality through the introduction
of a slack variable $\alpha > 0$,
\[ \sum_j x_{ij} + \alpha_i = b_i \tag{C2} \]

Likewise, the constraint on the demand for blend $j$ is written as,
\[ z_j = \sum_i x_{ij} \le d_j \]
which after introduction of the slack variable $\beta_j$ yields,
\[ \sum_i x_{ij} + \beta_j = d_j \tag{C3} \]

Lastly, we must ensure that each blend $j$ meets its minimum octane requirement,
\[ \sum_i o_i x_{ij} \ge r_j z_j \]
or equivalently,
\[ \sum_i o_i x_{ij} - r_j z_j - \gamma_j = 0 \tag{C4} \]

Our objective function can be formed from three distinct contributions.
The cost of purchasing the raw components is given by,
\[ F_1 = \sum_i p_i y_i \]
The revenue from selling the blended products is given by,
\[ F_2 = \sum_j p'_j z_j \]
The revenue from selling the excess raw components is given by,
\[ F_3 = \sum_i e_i \left(y_i - \sum_j x_{ij}\right) \]
The objective (profit) to be maximized is then given by,
\[ F = F_2 + F_3 - F_1 \tag{Obj} \]

We note that our optimization space now consists of vectors formed from
$(\vec x, \vec y, \vec z, \vec\alpha, \vec\beta, \vec\gamma) \ge 0$.

\section{Question 3}

We will need two projection methods:

\begin{enumerate}
\item A projection from a partially specified matrix symmetric matrix $M$
  to a PSD matrix $X \in S_+^n$.
\item A projection from a PSD matrix $X \in S_+^n$ to the set of
  completions of $M$.
\end{enumerate}

In both cases we will minimize the nuclear norm,
\[ \Vert X \Vert = \Vert V \Lambda V^T \Vert = \sum_i \sigma_i \]
where the singular value $\sigma_i = \vert \lambda_i \vert$.

For (a), we recognize that the symmetric matrix $X$ can be represented
in the eigenbasis of $M$, $X = V \Omega V^T$. However, $M$ may have
negative eigenvalues, which we must eliminate in order for $X$ to be
PSD. We recognize that in these cases $\Vert X \Vert$ is minimized by
letting $\omega_i = 0$ for all negative eigenvalues $\lambda_i <
0$.

For (b), we speculate that the minimal norm projection from $X$ to a
valid completion $M$ is $X$ with the observed entries placed.

This has been implemented in Python (see Figure
\ref{Fig:AltProj}). Convergence characteristics for a small,
underdetermined system is shown in Figure \ref{Fig:AltProjConv}.

\begin{figure}
  \lstinputlisting[language=Python]{q3.py}
  \caption{(Question 3) Implementation of method of alternating projections in Python.}
  \label{Fig:AltProj}
\end{figure}

\begin{figure}
  \center
  \includegraphics[scale=0.5]{q3-convergence.pdf}
  \caption{(Question 3) Convergence of method of alternating projections on 100x100
  symmetric matrix with 10 observed elements}
  \label{Fig:AltProjConv}
\end{figure}

\section{Question 4}

\begin{figure}
  \lstinputlisting[language=Python]{q4.py}
  \caption{(Question 4) Implementation of three linear optimization methods in Python.}
  \label{Fig:Q4}
\end{figure}

\begin{figure}
  \center
  \includegraphics[scale=0.5]{q4-convergence.pdf}
  \caption{(Question 4) Convergence of methods on 10 equation, 10 variable linear system}
  \label{Fig:Q4Convergence}
\end{figure}

\lstinputlisting[language=Python]{q3.py}

\section{Question 5}

\lstinputlisting[language=Haskell]{q5.hs}


\end{document}